\documentclass[frenchb]{article}
\usepackage[T1]{fontenc}
%Pour utilisation sous unix
\usepackage{a4wide}
\usepackage{graphicx}
\usepackage{amssymb}
\usepackage{color}
\usepackage{babel}
\usepackage[utf8]{inputenc}

\begin{document}

\begin{figure}[t]
\centering
\includegraphics[width=5cm]{inp_n7.png}
\end{figure}

\title{\vspace{4cm} {Réalisation d'un modem de fréquence selon la recommandation V21 de
l'Union Internationale des Télécommunications (UIT)}}
\author{Ayoub Bouchama \\ Oussama ElGuerraoui}
\date{\vspace{7cm} Département Sciences du Numérique - Première année \\
2022-2023 }

\maketitle

\newpage
\tableofcontents
\listoffigures

\newpage
\section{Introduction}
Le projet consiste en la réalisation d'un modem de fréquence, selon la recommandation V21 de l'Union Internationale des Télécommunications (UIT). Un modem (acronyme de modulateur/démodulateur) est un dispositif utilisé pour transformer une information numérique en un signal pouvant être transmis sur le support physique de communication entre l'émetteur et le récepteur (modulation), et pour retrouver l'information binaire à partir du signal altéré par le lien physique de communication (démodulation). Le modem à implanter dans ce projet sera utilisé sur la ligne téléphonique, avec un débit maximal de 300 bits par seconde (également exprimé en bauds). La technique de transmission utilisée est la modulation de fréquence numérique, qui est encore utilisée dans de nombreux systèmes de transmission actuels, tels que l'internet des objets (technologie LoRa), le Bluetooth ou la télémesure/télécommande par satellite. Le modem V21 fonctionne en mode duplex et simule la voie de transmission n°1 (appel). La première étape du projet consiste à créer le signal modulé en fréquence à partir d'un fichier d'information binaire. Ensuite, à partir d'un signal reçu altéré par un canal à bruit additif blanc et gaussien (AWGN), il faudra mettre en place le modem de réception permettant de retrouver l'information binaire. Plusieurs versions du récepteur seront implantées : une première version procédera par filtrage, une seconde utilisera la méthode de détection de corrélation de moyenne et une troisième mettra en œuvre une détection de corrélation de maximal. Enfin, la dernière étape consistera à évaluer la qualité de transmission obtenue en comparant l'information binaire transmise à celle reçue.

\section{Modem de fréquence}
    \subsection{Construction du signal modulé en fréquence} \label{sec:signal}
    La première étape du projet consiste à réaliser la modulation de fréquence, i.e. transformer l'information binaire à transmettre en un signal modulé en fréquence (exemple sur la figure \ref{fig : signalFSK}).\\
        \begin{figure}[ht!]
            \centering
            \includegraphics[width=8cm]{SignalFSK.png}
            \caption{Signal modulé en fréquence \label{fig : signalFSK}}
        \end{figure}
    
    Le signal modulé en fréquence $x(t)$ est généré de la manière suivante :
            \begin{equation} \label{signal_mod_freq}
            x(t)=(1-NRZ(t)) \times \cos(2 \pi F_0 t+\phi_0) + NRZ(t) \times \cos(2 \pi F_1 t+\phi_1)
            \end{equation}
    où $NRZ(t)$ est un signal de type NRZ polaire formé à partir de la suite de bits à transmettre en codant les $0$ et les $1$ par des niveaux $0$ et $1$ de durée $T_s$ secondes. $\phi_0$ et $\phi_1$ sont des variables aléatoires indépendantes uniformément réparties sur $\left[0, 2 \pi\right]$ qui peuvent être obtenues sous matlab en utilisant \emph{rand*2*pi}.

        \subsubsection{Génération du signal NRZ}
            \begin{itemize}
                \item Le signal NRZ (Non Return to Zero) est un type de signal numérique utilisé dans les communications de données pour représenter des bits binaires sous forme de tension électrique. Pour générer ce signal, nous utilisons la fonction randi() de Matlab pour générer un vecteur de bits aléatoires de longueur nbit. Nous utilisons ensuite la fonction kron() pour étendre chaque bit sur Ns échantillons consécutifs, de sorte que chaque bit du signal NRZ est représenté par un segment de Ns échantillons consécutifs de la même valeur
                \item Une fois le signal NRZ généré, nous pouvons l'afficher dans une échelle temporelle en utilisant la fonction plot() de Matlab. Nous pouvons également calculer et afficher sa densité spectrale en utilisant la fonction fft() pour calculer sa transformée de Fourier, puis en utilisant la fonction abs() pour obtenir sa magnitude et en la mettant au carré pour obtenir sa densité spectrale.
                \item Enfin, nous pouvons également calculer et afficher la densité spectrale du filtre idéal NRZ en utilisant le vecteur y de zéros, en mettant à 1 le coefficient correspondant à la fréquence 0 (dirac) et en utilisant la fonction sinc() pour définir la densité spectrale du filtre. Nous pouvons ensuite afficher les densités spectrales du signal NRZ et du filtre idéal NRZ sur une échelle logarithmique en utilisant la fonction semilogy()
            \end{itemize}
        
                
            

        \subsubsection{Génération du signal modulé en fréquence}
            \begin{itemize}
                \item Génération des porteuses : afin de moduler le signal NRZ, nous avons généré deux porteuses sinusoïdales de fréquences F0 = 6000 Hz et F1 = 2000 Hz. Nous avons également attribué une phase aléatoire à chacune de ces porteuses.
                \item Modulation du signal NRZ : nous avons modulé le signal NRZ en multipliant chaque bit par la porteuse correspondante (s0 pour les bits de valeur 0 et s1 pour les bits de valeur 1). Le résultat de cette opération est un signal modulé en fréquence, que nous avons appelé x.
                \item Affichage et analyse du signal modulé : nous avons affiché le signal x dans une échelle temporelle et calculé sa densité spectrale. Nous avons constaté que la densité spectrale du signal x présente des pics de puissance autour des fréquences F0 et F1, ce qui indique que le signal NRZ a bien été modulé en utilisant ces deux porteuses.
                \item Le calcul de la densité spectrale : \\
                 $$
                 x(t) = (1 - NRZ(t)) \, \cos (2\pi F_0 t + \phi_0) + NRZ(t)) \, \cos (\, 2\pi F_1 t + \phi_1)
                 $$
\\
Comme $ \phi_0 $ \ et \ $ \phi_1 $ sont des variables 
aléatoires indépendantes uniformément réparties sur [0, 2\pi] ,\ NRZ  \ signal \ aléatoire \ centrée.
\\ Alors : \\
\\
$ R_x(\tau) \ = \ $\text{E}[x(t) \ x*(t - \tau)] $ \\
$ \\ Donc : \\
 
\\$ R_x(\tau) = $ \text{E}[((1 - NRZ(t)) \, \cos (2\pi F_0 t + \phi_0) \ + \ NRZ(t) \, \cos (2\pi F_1 t + \phi_1)) \\ 
\\((1 - NRZ(t - \tau)) \, \cos (2\pi F_0( t - \tau) \ + \phi_0) \ + \ NRZ(t - \tau))\cos (2\pi F_1 ( t - \tau) + \phi_1))$]  \\
\\ = $ \text{E}[((1 - NRZ(t)) \, (1 - NRZ(t - \tau))] \text{E}[ \, \cos (2\pi F_0 t + \phi_0) \, \cos (2\pi F_0( t - \tau) + \phi_0)] \  +\\
 \\ \text{E}[((1 - NRZ(t)) \, NRZ(t - \tau)] \text{E}[ \, \cos (2\pi F_0 t + \phi_0) \, \cos (2\pi F_1( t - \tau) + \phi_1)] 
+\\
\\   \text{E}[(NRZ(t) \, (1 - NRZ(t - \tau))] \, \text{E}[ \, \cos (2\pi F_1 t + \phi_1)\, \cos (2\pi F_0( t - \tau) + \phi_0)] +\\
\\ 
 \text{E}[NRZ(t) \, NRZ(t - \tau)] \text{E}[\, \cos (2\pi F_1 t + \phi_1) \, \cos (2\pi F_1( t - \tau) + \phi_1)]\\
\\ =\text{E} [\, \cos (2\pi F_0 t + \phi_0) \, \cos (2\pi F_0( t - \tau) + \phi_0)] \ +  \text{E}[NRZ(t)\,NRZ(t - \tau)]\\
\\ \text{E}[\, \cos (2\pi F_0 t + \phi_0)\, \cos (2\pi F_0( t - \tau) + \phi_0)] \ + \ \text{E}[(NRZ(t) \, NRZ(t - \tau)]\, \text{E}[\, \cos (2\pi F_1 t + \phi_1)]\\
\\ \text{E}[ \, \cos (2\pi F_0( t - \tau) + \phi_0)] \ +  \ \text{E}[NRZ(t)\, NRZ(t - \tau)] \text{E}[\, \cos (2\pi F_0 t + \phi_0)] \text{E}[ \, \cos (2\pi F_1( t - \tau) + \phi_1)]\ +\\
\\ \text{E}[NRZ(t ) \, NRZ(t - \tau)]\text{E}[\, \cos (2\pi F_1 t + \phi_1) \, \cos (2\pi F_1( t - \tau) + \phi_1)] \\
\\R_x(\tau) \  = \frac{R_{NRZ}(\tau)}{2} \ [\, \cos(2\pi F_0 \tau) \ +\  \cos(2\pi F_1 \tau) \,]$ \\ \
\\ 
\\ car : \\ 
$\\ \text{E}[ \cos (2\pi F_0 t + \phi_0)] = \text{E}[ \cos (2\pi F_1 t + \phi_1)] = 0\\
\\ \text{E} [ \cos (2\pi F_0 t + \phi_0)\cos (2\pi F_0( t - \tau) + \phi_0)] = \cos(2\pi F_0 \tau) \\
\\ \text{E} [ \cos (2\pi F_1 t + \phi_0)\cos (2\pi F_1( t - \tau) + \phi_0)] =\cos(2\pi F_1 \tau) $
\\
\\

La densité spectrale de puissance :
\\

$ S_x(f) \ = \ TF(R_x(\tau) \\
\\ = \ \frac{1}{2} TF(\ R_{NRZ}(\tau)\ ) \ *\ TF(\ \cos(2\pi F_0 \tau) \ +\  \cos(2\pi F_1 \tau)\ )\\
\\ = \ \frac{1}{2} S_{NRZ}(f)\ * \ (\delta(f \ - \ f_1) \ + \ \delta(f \ - \ f_0)) \\
\\ S_x(f) \ = \ \frac{1}{2} (S_{NRZ}(f-f0) \ + \ S_{NRZ}(f-f1) )

\newpage
\section{Canal de transmission à bruit additif, blanc et Gaussien} \label{sec:canal}
    Nous allons considérer que le canal de propagation ajoute au signal émis un bruit que l'on suppose blanc et Gaussien et qui modélise les perturbations introduites. \\
    La puissance du bruit Gaussien à ajouter devra être déduite du rapport signal sur bruit (SNR : Signal to Noise Ratio) souhaité pour la transmission donné en dB :
    $$
    SNR_{dB}=10 \log_{10} \frac{P_x}{P_b}
    $$
    où $P_x$ représente la puissance du signal modulé en fréquence et $P_b$ la puissance du bruit ajouté.

\section{Démodulation par filtrage}
    La figure \ref{fig : demod_filtrage} présente le récepteur implanté pour retrouver, à partir du signal modulé en fréquence bruité, le message binaire envoyé.

    \begin{figure}[ht!]
        \centering
        \includegraphics[width=14cm]{demod_filtrage.png}
        \caption{Démodulation par filtrage. \label{fig : demod_filtrage}}
     \end{figure}

    Un filtre passe-bas permet de filtrer les morceaux de cosinus à la fréquence $F_0=6000$Hz, tandis qu'un filtre passe-haut permet de filtrer les morceaux de cosinus à la fréquence $F_1=2000$Hz. Une détection d'énergie réalisée tous les $T_s$ secondes permet de récupérer, à partir des signaux filtrés, les bits $0$ et $1$ transmis.\\

        \subsection{Synthèse du filtre passe-bas}
                La fréquence de coupure du filtre passe-bas est définie comme la fréquence moyenne entre les fréquences de la porteuse 0 et de la porteuse 1, c'est-à-dire (F0 + F1) / 2.
                
                Un vecteur k compris entre -N et N est créé, où N est un entier déterminant la longueur du filtre.
                
                L'intervalle de temps entre chaque échantillon du signal filtré est calculé en divisant l'inverse de la fréquence d'échantillonnage par le nombre d'échantillons.
                
                La fréquence normalisée du filtre passe-bas est calculée en divisant la fréquence de coupure par la fréquence d'échantillonnage.
                
                Le vecteur de la réponse impulsionnelle du filtre passe-bas est créé en utilisant la fonction sinc et en multipliant chaque élément par 2 * f-tilde.
                
                Un vecteur de temps compris entre -N * Te et N * Te est créé en utilisant l'intervalle de temps calculé précédemment.
                
                La réponse impulsionnelle du filtre passe-bas est tracée sur un graphique en utilisant le vecteur de temps comme abscisses et le vecteur de la réponse impulsionnelle comme ordonnées.
                
                La transformée de Fourier de la réponse impulsionnelle du filtre passe-bas est calculée et décalée.
                
                Un vecteur de fréquences compris entre -Fe/2 et Fe/2 est créé en utilisant l'intervalle de fréquences correspondant à chaque élément du vecteur de la transformée de Fourier.
                
                La réponse en fréquence du filtre passe-bas est tracée sur un graphique en utilisant le vecteur de fréquences comme abscisses et la valeur absolue du vecteur de la transformée de Fourier comme ordonnées.

        \subsection{Synthèse du filtre passe-haut}
        Dans cette section, nous avons synthétisé un filtre passe-haut (PH) en utilisant la réponse impulsionnelle du filtre passe-bas (PB) précédemment déterminée. Pour ce faire, nous avons simplement utilisé la formule suivante : signal-filtre-PH = dirac1(k*Te)-signal-filtre-PB.

Nous avons ensuite affiché la réponse impulsionnelle du filtre PH en utilisant la commande plot et en spécifiant le vecteur des temps temps et le vecteur de la réponse impulsionnelle signal-filtre-PH. Nous avons également affiché la réponse en fréquence du filtre PH en utilisant la commande plot et en spécifiant le vecteur des fréquences F2 et le vecteur de la réponse en fréquence fft-filtre-PH.

Ces résultats nous permettent de visualiser la réponse impulsionnelle et la réponse en fréquence du filtre PH synthétisé, ce qui est essentiel pour la suite de l'étude du modem de fréquence.

        \subsection{Résultats obtenus avec un ordre des filtres de $61$}
        A COMPLETER

        \subsection{Modification de l'ordre des filtres}
        A COMPLETER
        
        \subsection{Utilisation des fréquences de la recommandation V21}
        A COMPLETER
        
\section{Démodulateur de fréquence adapté à la norme V21}
    \subsection{Contexte de synchronisation idéale}
    La figure \ref{fig : demod_FSK} présente le récepteur implanté afin de retrouver, dans un contexte de synchronisation idéale, le message binaire envoyé à partir du signal modulé en fréquence suivant la recommandation V$21$.   
    \begin{figure}[ht!]
        \centering
        \includegraphics[width=12cm]{demod_FSK1.png}
        \caption{Démodulation FSK. Synchronisation supposée idéale. \label{fig : demod_FSK}}
     \end{figure}

    \subsubsection{Principe de fonctionnement de ce récepteur}
    A COMPLETER
    
    \subsubsection{Résultats obtenus}
    A COMPLETER


    \subsection{Gestion d'une erreur de synchronisation de phase porteuse}
    Le problème de la synchronisation entre l'émetteur et le récepteur est un problème important lorsque l'on réalise une transmission. Les deux doivent être parfaitement synchronisés en temps et en fréquence pour que le démodulateur implanté précédemment fonctionne, ce qui en pratique n'est bien entendu pas possible.
    Afin que le modem puisse continuer à fonctionner en présence d'une erreur de phase porteuse, celui-ci doit être modifié. La figure \ref{fig : demod_FSK2} présente un démodulateur permettant de s'affranchir de problèmes de synchronisation de phase entre les oscillateurs d'émission et de réception.
        \begin{figure}[ht!]
    		\centering
    		\includegraphics[width=14cm]{demod_FSK2.png}
    		\caption{Démodulation FSK - Gestion d'une erreur de phase porteuse. 
    		\label{fig : demod_FSK2}}
 		\end{figure}

    \subsubsection{Impact d'une erreur de phase porteuse sur le modem implanté précédemment}
    A COMPLETER
    
    \subsubsection{Principe de fonctionnement de cette nouvelle version du modem de fréquence}
    A COMPLETER (comment ce nouveau démodulateur permet il de gérer une erreur de phase porteuse)
    
    \subsubsection{Résultats obtenus}
    A COMPLETER

\section{Conclusion}
A COMPLETER

\section{Références}
A COMPLETER

\end{document}